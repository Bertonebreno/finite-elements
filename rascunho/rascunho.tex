\documentclass{article}
\usepackage{amsmath}
\usepackage{amsfonts}

\begin{document}

\begin{itemize}
    \item[(a)] Encontre a equação de Euler-Lagrange do Problema V e determine sua solução analítica considerando \(\gamma = 1\).
\end{itemize}

\textbf{Resposta:}

Para encontrar a equação de Euler-Lagrange do Problema V, consideramos a formulação variacional:
\[
\int_0^1 (\alpha u'v' + u v) \, dx = \int_0^1 v \, dx \quad \forall v \in \mathcal{U},
\]
onde \(\mathcal{U} = H^1_0(0,1)\).

A equação de Euler-Lagrange associada a este problema é obtida aplicando o método das variações. O funcional associado é:
\[
J[u] = \frac{1}{2} \int_0^1 (\alpha u'^2 + u^2) \, dx - \int_0^1 u \, dx.
\]

A equação de Euler-Lagrange para um funcional da forma:
\[
J[u] = \int_a^b L(x, u, u') \, dx
\]
é dada por:
\[
\frac{\partial L}{\partial u} - \frac{d}{dx} \left( \frac{\partial L}{\partial u'} \right) = 0.
\]

Para o nosso problema, o Lagrangiano \( L \) é:
\[
L(u, u') = \frac{1}{2} (\alpha u'^2 + u^2) - u.
\]

Calculamos as derivadas parciais:
\[
\frac{\partial L}{\partial u} = u - 1,
\]
\[
\frac{\partial L}{\partial u'} = \alpha u'.
\]

Calculamos a derivada total com respeito a \(x\):
\[
\frac{d}{dx} \left( \frac{\partial L}{\partial u'} \right) = \alpha u''.
\]

Substituímos na equação de Euler-Lagrange:
\[
u - 1 - \alpha u'' = 0,
\]
ou seja,
\[
\alpha u'' - u = -1.
\]

Portanto, a equação de Euler-Lagrange é:
\[
\alpha u'' - u = -1.
\]

 Solução Analítica

Para resolver esta equação diferencial, consideramos a equação homogênea associada:
\[
\alpha u'' - u = 0.
\]

A solução geral da equação homogênea é:
\[
u_h(x) = C_1 e^{\sqrt{\frac{1}{\alpha}} x} + C_2 e^{-\sqrt{\frac{1}{\alpha}} x}.
\]

Para a equação completa, procuramos uma solução particular \( u_p \):
\[
\alpha u_p'' - u_p = -1.
\]

Tentamos uma solução particular constante \( u_p = A \):
\[
-A = -1 \quad \Rightarrow \quad A = 1.
\]

Portanto, a solução geral da equação diferencial é:
\[
u(x) = C_1 e^{\sqrt{\frac{1}{\alpha}} x} + C_2 e^{-\sqrt{\frac{1}{\alpha}} x} + 1.
\]

Aplicamos as condições de contorno \( u(0) = 0 \) e \( u(1) = 0 \):

1. Para \( u(0) = 0 \):
\[
C_1 + C_2 + 1 = 0 \quad \Rightarrow \quad C_1 + C_2 = -1.
\]

2. Para \( u(1) = 0 \):
\[
C_1 e^{\sqrt{\frac{1}{\alpha}}} + C_2 e^{-\sqrt{\frac{1}{\alpha}}} + 1 = 0.
\]

Resolvendo este sistema de equações lineares para \( C_1 \) e \( C_2 \):

\[
\begin{cases}
C_1 + C_2 = -1, \\
C_1 e^{\sqrt{\frac{1}{\alpha}}} + C_2 e^{-\sqrt{\frac{1}{\alpha}}} = -1.
\end{cases}
\]

Vamos resolver este sistema:

\[
C_1 = \frac{-1 - e^{-\sqrt{\frac{1}{\alpha}}} (-1)}{e^{\sqrt{\frac{1}{\alpha}}} - e^{-\sqrt{\frac{1}{\alpha}}}} = -\frac{1}{e^{\sqrt{\frac{1}{\alpha}}} - e^{-\sqrt{\frac{1}{\alpha}}}},
\]
\[
C_2 = -1 - C_1 = -1 + \frac{1}{e^{\sqrt{\frac{1}{\alpha}}} - e^{-\sqrt{\frac{1}{\alpha}}}}.
\]

Portanto, a solução é:

\[
u(x) = -\frac{1}{e^{\sqrt{\frac{1}{\alpha}}} - e^{-\sqrt{\frac{1}{\alpha}}}} e^{\sqrt{\frac{1}{\alpha}} x} + \left(-1 + \frac{1}{e^{\sqrt{\frac{1}{\alpha}}} - e^{-\sqrt{\frac{1}{\alpha}}}}\right) e^{-\sqrt{\frac{1}{\alpha}} x} + 1.
\]

\end{document}